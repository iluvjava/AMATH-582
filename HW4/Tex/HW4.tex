\documentclass{article}

% If you're new to LaTeX, here's some short tutorials:
% https://www.overleaf.com/learn/latex/Learn_LaTeX_in_30_minutes
% https://en.wikibooks.org/wiki/LaTeX/Basics

% Formatting
\usepackage[utf8]{inputenc}
\usepackage[margin=1in]{geometry}
\usepackage[titletoc,title]{appendix}

% Math
% https://www.overleaf.com/learn/latex/Mathematical_expressions
% https://en.wikibooks.org/wiki/LaTeX/Mathematics
\usepackage{amsmath,amsfonts,amssymb,mathtools}

% Images
% https://www.overleaf.com/learn/latex/Inserting_Images
% https://en.wikibooks.org/wiki/LaTeX/Floats,_Figures_and_Captions
\usepackage{graphicx,float}

% Tables
% https://www.overleaf.com/learn/latex/Tables
% https://en.wikibooks.org/wiki/LaTeX/Tables

% Algorithms
% https://www.overleaf.com/learn/latex/algorithms
% https://en.wikibooks.org/wiki/LaTeX/Algorithms
\usepackage[ruled,vlined]{algorithm2e}
\usepackage{algorithmic}

% Code syntax highlighting
% https://www.overleaf.com/learn/latex/Code_Highlighting_with_minted
% \usepackage{minted}
%\usemintedstyle{borland}

% References
% https://www.overleaf.com/learn/latex/Bibliography_management_in_LaTeX
% https://en.wikibooks.org/wiki/LaTeX/Bibliography_Management
\usepackage{biblatex}
\addbibresource{references.bib}

\usepackage{hyperref}
\hypersetup{
    colorlinks=true,
    linkcolor=blue,
    filecolor=magenta,      
    urlcolor=cyan,
}

% Title content
\title{AMATH 582 HW 4: SVM, LDA, And Decision Trees on MNIST}
\author{Hongda Li}
\date{\today}

\begin{document}

\maketitle

% Abstract
\begin{abstract}
   The objective is to develope basic practical and theoretical understandings of some classical machine learning methods. We use the MNIST data set and the models are Linear Discriminant Analysis, Principal Component Analysis, Support Vector Machine, and Binary Decision Tree. Matlab is the major tool. PCA is used as a dimensionality reduction technique for the data. Models are trained on the projection onto the principal component mode to increase efficiency. Models are trained to distinguished pair of of digits, or all classifier all of the digits. Metrics used are the errors between the predicted labels and the actual labels and the confusion matrix. The metrics is measure on training, validations, and test set. Theory and intuitions behind the algorithms and practical details are explored in the paper. 
\end{abstract}


\section{Introduction and Overview}
    \hspace{1.1em}
    We were given the Mnist data set, which are images of hand written digits, packed into tensor of size $28 \times 28 \times N$. 60k of them are used for training, and 10k of them are used for testing. Samples for each digits are approximately homogenous. 
    \par
    The classification that we are interested in are SVM: Support Vector Machine, LDA: Linear Discriminant Analysis, and Decision Tree (DTree). 
    \par
    Principal Component Analysis (PCA) is used as a dimensionality reduction technique for the data set. More information on the theory part. 
    \par
    Confusion matrix is used as a performance metrics for all above classifications. And this metrics is being measured for training, validations, and the test set. It's used because it exposes more information than the errors. 
    \par
    In addition, visualization will be made for PCA to demonstrate the low rank property of the data set. 

\section{Theoretical Background}
    \hspace{1.1em}
    Most of the ideas discussed below are referencing sources from Mathworks, Sklearn, lecture notes of Math 253 from San Jose State University, The \textit{Book Elements of Statistical Learning} from Springer by Trevor Hastie et al, and the book \textit{Data-Drive Modeling \& Scientific Computation} from the University of Washington by Nathan Kutz. 
    \subsection{Linear Discriminant Analysis}
        \hspace{1.1em}
        Linear discriminant analysis comes in many different types of variance, and it can be regularized as well. The most simple is: LDA, which seeks to look for a linear subspace such that the projection of the labeled data onto the line has maximal distance from the global central, and minimal variance within the clusters. The model assume homogenous Gaussian distribution of the data points. If Non-homogenous Gaussian clusters are involved, quadratic discriminant analysis is needed for the classifier. However if QDA classifier is used, then the covariance matrix for the features for each cluster will have to be diagonal, implying the fact that projecting onto the PCA modes before the analysis will be helpful for QDA.  
        \par
        WLOG, let's assume a binary classification problems with 2 classes: $C_1, C_2$. The features of each sample is a vector denoted by $x_i$ and the subspace we project them onto is a vector denoted as $v$, then, the centroid for one of the cluster projected onto the line can be expressed as\cite{lda}:
        \begin{equation*}\tag{1}\label{eqn:1}
            \mu_1 = \frac{1}{|C_1|} \sum_{x_i \in C_1}^{}\left(
                v^Tx_i
            \right) = \frac{v^T}{|C_1|} \sum_{x_i\in C_1}^{}\left(
                x_i
            \right)
        \end{equation*}
        And conveniently, the centroid of the cluster can be expressed as: 
        \par
        \begin{equation*}\tag{2}\label{eqn:2}
            m_1 = \frac{1}{|C_1|} \sum_{x_i\in C_1}^{}\left(
                x_i
            \right)
        \end{equation*}
        Consider the distance between $\mu_1, \mu_2$ on the projected space, which can be expressed as: 
        \begin{align*}\tag{3}\label{eqn:3}
            (\mu_1 - \mu_2)^2 &= (v^Tm_1 - v^Tm_2)^2 = (v^T(m_1 - m_2))^2 
            \\
            (v^T(m_1 - m_2))^2  &= v^T(m_1 - m_2)\cdot (m_1 - m_2)^Tv
            \\
            S_\text{between} &= (m_1 - m_2)(m_1 - m_2)^T\in \mathbb{R}^{d\times d}
        \end{align*}
        Where we have define the rank-one matrix that measures the distance between the projected mean of the clusters. 
        \par
        Now consider another matrix which will be helpful for us: WLOG, the variance for the class $C_1$ can be computed by: 
        \begin{align*}\tag{4}\label{eqn:4}
            s_1^2 &= \sum_{x_i \in C_1}^{}\left(
                    (v^Tx_i - v^Tm_2)^2
                \right)
                =
                \sum_{x_i\in C_1}^{}\left(
                        v^T(x_i - m_1)(x_i - m_1)^Tv
                    \right)
            v^T
            \\
            s_1^2 &=
            \left\lbrack
                \sum_{x_i\in C_1}^{}\left(
                        (x_i - m_1)(x_i - m_2)
                    \right)
            \right\rbrack v
            = v^TS_{\text{within}}v
        \end{align*}
        Then, the problem of maximizing the centroid of the cluster onto the sub-space and minimizing all the variance of the cluster can be summarized as: 
        \begin{align*}\tag{5}\label{eqn:5}
            \max_{\left\Vert
                v
            \right\Vert = 1} \left\lbrace
                \frac{(\mu_1 - \mu_2)^2}{s_1^2 + s_2^2}
            \right\rbrace
            =
            \max_{\Vert v\Vert = 1} \left\lbrace
            \frac{v^TS_{\text{between}}v}{v^TS_{\text{within}}v}
            \right\rbrace
        \end{align*}
        And computationally, this become an augmented Rayleigh Quotient, which can be solved by looking for vector in the Eigenspace of the matrix $S_{\text{between}}S_{\text{within}}^{-1}$. 
        \par
        Take notice that, this formation is from an optimization point of view instead of using Bayes Theorem, and using the idea of Bayes theorem, one can formulate the problem for Quadratic Discriminant Analysis and derive the computational problem. 
        \par
        Take note that, for multi-class, LDA take the squared distance between each cluster from the global centroid of all the data point as a measure for the spread of the clusters. 
        \par
        In addition, we need to notice that LDA can be used both as a dimensionality reduction technique and a classifier. The Eigenspace of the matrix $S_{\text{between}}S_{\text{within}}^{-1}$ is subspace which, maximal separation between the data occurred. However, it doesn't have to be an orthogonal subspace. 

    \subsection{Principal Component Analysis}
        \hspace{1.1em}
        Principal Component Analysis is used as a dimensionality reduction technique\cite{kutz_2013_387}. Consider the Column Data Matrix $X$ and its SVD decomposition of the matrix: 
        \begin{equation*}\tag{6}\label{eqn:6}
            X = U\Sigma V^T \implies U^TX = \Sigma V^T
        \end{equation*}
        Where: $U$, $V$ are both orthogonal matrices. Assuming that the matrix $X$ is $m\times n$ and in the case of our data set, $m << n$, then the matrix $U\in \mathbb{R}^{m\times m}$, $\Sigma \in\mathbb{R}^{m\times m}$, and $V^T\in\mathbb{R}^{m\times n}$. 
        \par
        The Interpretation of SVD is simple. We decompose the column data matrix into Principal Components that explains the most variance (Columns of the matrix $U$) and the singular value, which explain the importance of each mode (Diagonals of matrix $\Sigma$), and the profile vector, which describes the composition for each of the sample (Column vector of the matrix $V^T$, rows of the matrix $V$). 
        \par
        Conveniently, the principal components of the data matrix is ordered with the singular value and singular values are ordered in descending order. 
        \par
        This means that, we can truncate the rows of the matrix $\Sigma V^T$ yet still retaining most of the information for each column of the data matrix. This is the ``Eckart-Young" Theorem, a major idea behind PCA and dimensionality reduction. Assuming that we now truncate the SVD into using matrices of the size $m \times k, k\times k, k\times n$ for $U, \Sigma, V^T$ where $k << m$ then it also has the added benefit that the sample spans all the subspace $\mathbb{R}^k$, implying that the covariance matrices for the classes are going to be non-singular. Which is a requirement for Quadratic Discriminant Analysis. 
        
    \subsection{Support Vector Machine}
        \hspace{1.1em}
        The idea of SVM is to find the hyper plane between the clusters such that it maximizes the margin between the boundary of the 2 clusters. This is intrinsically a binary classification method\cite{scikit-learn}. 
        % Multi-class reference: https://scikit-learn.org/stable/modules/multiclass.html
        \par
        Computationally the problem is phrased as a Quadratic Programming problem. The SVM is computationally phrased as\cite{stats_book_419}: 
        \begin{equation*}\tag{7}\label{eqn:7}
            \min_{\beta, \beta_0} \left\Vert
                 \beta
            \right\Vert \text{ s.t: } 
            \begin{cases}
                y_i(x_i^T\beta + \beta_0) \ge 1 - \xi_i \quad \forall i
                \\
                \xi_i \ge 0, \sum_{i}^{}\left(\xi_i\right) \le \text{constant}
            \end{cases}
        \end{equation*}
        Where $y_i\in \{1, -1\}$ are the binary label of the sample, and $\xi_i$ is the penalty for cross the boundary for each of the support vector, and $\beta$ is the vector the defines the hyperplane. 
        % Reference from elements of statistical learning: page 419, eqn 12.7
        \par
        In practice, to get the best separations for different topology of the cluster, kernel function is often used as a way of mapping the same data from a lower dimension to a high dimension, while the theoretical analysis of the dual problem allows improve the speed and make use of only the pair-wise inner product under the new basis\cite{stats_book_424}. 

    \subsection{Binary Decision Tree}
        \hspace{1.1em}
        The idea of binary tree is a simple yet powerful concepts. A tree is created to partition the feature space on each node of the tree. Each branching of the tree assert one conditions on one of the attribute, (e.g: age older than 18), and each leaf node consists of samples with almost the same labels. 
        \par
        There are many variance of the decision tree, but for our purposes, we will be focusing on the default that comes with the ``fictree'' command in matlab\cite{MathWorks-fitctree}. 
        \par
        Decision tree unlike method described above, doesn't have smooth decision boundary, making it sensitive to noise and extrapolation of the data. However, it's it has a variety of hyper parameters available for tunning and it's a interpretable\cite{steven}.

\section{Algorithm Implementation and Development}
    \hspace{1.1em}
    The challenging part of machine learning practice in this case is data preprocessing and dimensionality reduction of the data. Hence for this part we will be highlighting the procedures for PCA, unit variance zero mean Standardization, and MinMax Standardization, and the generic training process of various models. 
    \par
    Others parts of the practical implementations that are not listed here involves the procedures of visualizing the results, setting up the parameters, and filtering out certain digits from the training set. 
    \subsection{Data Standardization}
        \begin{algorithm}\label{alg:algo1}
            \begin{algorithmic}[H]\algsetup{linenosize=\tiny}\scriptsize
                \STATE{\textbf{Input: Data}: D }
                \STATE{D := Flattten the Tensor: D into a column data matrix, convert the data type as well.}
                \STATE{D := D - mean(D, 1)}
                \STATE{D := D./std(D, 1)}
                \STATE{\textbf{Return:} D}
            \end{algorithmic}\caption{Algorithm 1: Zero Mean Unit Variance Standardization}
        \end{algorithm}
        \begin{algorithm}\label{alg:algo2}\algsetup{linenosize=\tiny}\scriptsize
            \begin{algorithmic}[H]
                \STATE{\textbf{Input: Data} D}
                \STATE{\textbf{Precondition:} \hyperref[alg:algo1]{Algorithm 1} Has been run on D. }
                \STATE{D := data - minAll(D)}
                \STATE{D := data./maxAll(D)}
                \STATE{\textbf{Return:} D}
            \end{algorithmic}\caption{Algorithm 2: MinMax Standardization}
        \end{algorithm}
        \par
        There are 2 standardization techniques used in the training process of various models. The zero mean unit variance standardization(\hyperref[alg:algo1]{Algorithm 1}), which is used for the PCA, and the LDA. The MinMax standardization(\hyperref[alg:algo2]{Algorithm 2}), which is used for the SVM. The SVM only works for points that have relative distance due to the penalty of the corresponding optimization problem, hence MinMax standardization is needed.
    \subsection{PCA Dimensionality Reduction}
        \begin{algorithm}\label{alg:algo3}\algsetup{linenosize=\tiny}\scriptsize
            \begin{algorithmic}[H]
            \STATE{\textbf{Input}: \textbf{Data}: D \textbf{EnergyLevel}: E} 
            \STATE{\textbf{Preconditions:} The data, D is processed by \hyperref[alg:algo1]{Algorithm 1}}
                \STATE{Perform SVD on dat matrix D to obtain matrices: S, Sigma, V}
                \STATE{EnergyS := cumsum(Sigma)/sumAll(Sigma)}
                \STATE{Idx := the index of first number in EnergyS such that it's $>$ E}
                \STATE{ProjData := $\text{Sigma}*\text{V}^T$}
                \STATE{ProjData := ProjData(1: Idx, :)}
                \STATE{Projector := $\text{U(1: Idx, :)}^T$}
                \STATE{\textbf{Return:} ProjData, Projector}
            \end{algorithmic}
            \caption{Algorithm 3: PCA Dimensionality Reduction}
        \end{algorithm}
        \hyperref[alg:algo3]{Algorightm 3} describes the procedure for PCA dimensionality reduction. Variable ``D'' is the data matrix of the images, where each images are packed as columns of the matrix D, and ``E'' is the cumulative energy level we want for all the modes that we are projecting onto. The function then, returns the data described using all first ``Idx'' principal modes, and return the projector, which is just the matrix $\text{U}^T$. Note, the function ``sumAll'' does sum up all the elements in the tensor and it's out put is a scalar. 
    \subsection{Training Generic Models}
        \begin{algorithm}\label{alg:algo4}\algsetup{linenosize=\tiny}\scriptsize
            \begin{algorithmic}[1]
                \STATE{\textbf{Input: Training Data and Labels:} TrD,TrL \textbf{Testing Data and Labels: }TD, TL}
                \STATE{Standardize the data: TrD, TrL, TD, and TL appropriate to the model}
                \STATE{X, X1, Y, Y1 := Permuate the trainig data, 10\% for corss validation, 90\% for training using TrD, TrL}
                \STATE{Train the model using X, Y}
                \STATE{Produce labels using model on X, Y, X1, Y1}
                \STATE{Produce labels using model on TD, TL}
                \STATE{\textbf{Return:} All the label produced, data set and the model itself}
            \end{algorithmic}\caption{Algorithm 4: Generic Model Training Process}
        \end{algorithm}
        \hyperref[alg:algo4]{Algorithm 4} refers to the generic procedures of producing a model given the training and test MNIST Data set. Which is important for checking the constancy of the cross validation and the training set, and it's then benchmarked on the test set to see if there are potential for over-fitting. 
        \par
        For our purpose, we trained SVM, LDA, and Decision Tree using the Gaussian Kernel function for SVM, linear discriminant for LDA and they are all trained on the first 66 principal mode which covers just over 50\% of the cumulative energy on the singular value spectrum. The choice of using $50\%$ of the energy is aimed for reducing noise in the data, which will help with models that are sensitive to noise such as the SVM and the decision tree. When training the SVM model, all the data point is normalized using \hyperref[alg:algo2]{Algorithm 2} to transform the features such that the feature lies in $[0, 1]^66$.
 
\section{Computational Results}
    \subsection{Visualizing Low-rank Structure Using PCA}
        \begin{figure}[h]
            \hspace{-4em}
            \includegraphics*[width=0.6\linewidth]{singular-value-spectrum.png}
            \hspace{-2em}
            \includegraphics*[width=0.6\linewidth]{3-mode-projection.png}
            \caption{Figure 1: PCA Project and Singular Spectrum}
            \label{fig:fig1}
        \end{figure}
        \hspace{1.1em}
        The low-rank structure of the MNIST data is exposed by the decay of the singular value. Shown on left side of \hyperref[fig:fig1]{Fig 1}. In addition, some form of clustering on the mode 1, 5, 8 is shown on the right side of \hyperref[fig:fig1]{Fig 1} where each color represents a digit, and is having their own approximate region in the projected space. 66 principal modes are used for training all the models model, based on the assumption that most of the variances are noises. 
    \subsection{Performance Comparison on 2,3 Digits Separation}
        \hspace{1.1em}
        The strategy for looking for the hardest digits to separate is referencing the confusion matrix across the test set of the LDA method on all 10 digits, this is shown in \hyperref[fig:figure5]{Fig 5}. This is achieved by training all the digits on the LDA model on a one vs one setting. Focusing on the misclassified rate for each pairs of digits, it's not hard to see that the digits $4, 9$ are the digits where the model make most of the mistakes. And the digits that the model makes the least amount of mistakes are the digits: $1, 0$. Shown in \hyperref[fig:figure2]{Fig 2} the LDA misclassified $13.4\%$ of the instances of $9$ as $4$ and $1.5\%$ instances of $4$ as $9$ on the test set, but sightly lower on the training set and validation set, indicating a potential for \textbf{over fitting}. Between the easiest digits: 0, 1, the LDA models misclassified $0.2\%$ of the digits $0$ as $1$ and $0.1\%$ of the digits $1$ as $0$ on the test set (\hyperref[fig:figure3]{Fig 3}). Training error and the test error is extremely close, indicating a very good separations of the distribution between these 2 classes on the PCA modes. LDA shows some efforts when it tries to separate 3 digits, and the confusion matrix is shown under \hyperref[fig:figure4]{Fig 4}
        \par
        SVM performs better than the LDA method on binary classification of the digits pair $1, 0$ and $4, 9$ and the kernel function is set to ``Gaussian''. 0.9\% of the digit 4 is misclassified as $9$ and only 3.9\% of the digits 9 is misclassified as $4$ on the test set, shown in \hyperref[fig:figure7]{Fig 7}. For the separation between $0, 1$, non of the zero is classified as 1 and only 0.2\% instances of $1$ is mis-classified(\hyperref[fig:figure6]{Fig 6}). The SVM performs better compare to the LDA methods in both regards.
        \par
        The Decision tree method performs as good as the SVM models on the $4, 9$ separation but worse on the $0, 1$ separation, this is shown in \hyperref[fig:figure9]{Fig 9} and \hyperref[fig:figure10]{Fig 10}. The DTree misclassified $0.4\%$ and $0.7\%$ of the ``0'' as ``1" and ``1'' as ``0'' on the test set. For $4, 9$, it misclassified $13.9\%$ of 4 as 9 and $14.6\%$ of 9 as 4, higher than training set, indicating a potential \textbf{over fit} for the decision tree. 

    \subsection{Performance Comparison on 10 Digits Classifications}
        Two models that has been benchmarked on all 10 digits are the SVM and the LDA which is shown in \hyperref[fig:figure5]{Fig 5} and \hyperref[fig:figure8]{fig 8}.  Additional testing in the source code indicates that the QDA classifier performs as well as the SVM model. One explanation is that the distribution of samples of each class doesn't have the same amount of variance because LDA is not suited for classes with different amount of variance. Finally, the worst performance of LDA is due to under fitting because it has the same performance across all 3 of the test set and the training set. 

\section{Summary and Conclusions}
        \hspace{1.1em}
        The performance improvement on the accuracy and speed for using PCA as a dimensionality reduction technique is drastic. It's implement as extra in the source code, which we demonstrated that the decision tree become useless if the raw digits inputs are use to train the model. One can surmise the improvement if we made the choice of using LDA to find the best subspace on the data projected onto the PCA.  
        \par
        In addition, the importance to rescale the data into the $[0, 1]^n$ unit box in the positive quadrant for SVM is crucial in practice. This is required or the SVM to compute efficiently with a accuracy. 
        \par
        The decision tree model performs decently well only if it's trained on the projection onto the principal modes, which reflects the fact that Decision tree is sensitive to noise and pron to over-fit. To overcome this, one can consider using random forest to improve the accuracy, or Regression tree with a dimensionality reduction technique such as the PCA, or the LDA. 
   
% References
\printbibliography

% Appendices
\begin{appendices}
    \begin{figure}[h]
        \includegraphics*[width=1\linewidth]{conmat-lda-2-hardests-digits-pca.png}
        \caption{Figure 2: Hardest 2 Digits}\label{fig:figure2}
        \includegraphics*[width=1\linewidth]{conmat-lda-2-easiest-digits-pca.png}
        \caption{Figure 3: Easiest 2 Digits} \label{fig:figure3}
        \includegraphics*[width=1\linewidth]{conmat-lda-digits-with-pca.png}
        \caption{Figure 4: 3 Digits LDA}\label{fig:figure4}
        \includegraphics*[width=1\linewidth]{conmat-lda-alldigits-with-pca.png}
        \caption{Figure 5: All Digits LDA}\label{fig:figure5}
        \label{fig:fig2}
    \end{figure}
    \begin{figure}[h]
        \includegraphics*[width=1\linewidth]{conmat-svm-easiest-2-digits-pca.png}
        \caption{Figure 6: SVM on Easier 2 Digits}\label{fig:figure6}
        \includegraphics*[width=1\linewidth]{conmat-svm-hardest-2-digits-pca.png}
        \caption{Figure 7: SVM on hardest 2 Digits}\label{fig:figure7}
        \includegraphics[width=1\linewidth]{conmat-svm-alldigits-pca.png}
        \caption{Figure 8: SVM on all 10 Digits}\label{fig:figure8}
        \label{fig:fig3}
    \end{figure}
    \begin{figure}[h]
        \includegraphics*[width=1.1\linewidth]{conmat-dtree-easiest-2-digits.png}
        \caption{Figure 9: DTree on Easier 2 Digits}\label{fig:figure9}
        \includegraphics*[width=1.1\linewidth]{conmat-dtree-hardest-2-digits.png}
        \caption{Figure 10: DTree on Hardest 2 Digits}\label{fig:figure10}
    \end{figure}

\section{MATLAB Functions}

\end{appendices}

\end{document}
