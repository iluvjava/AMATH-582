\documentclass{article}

% If you're new to LaTeX, here's some short tutorials:
% https://www.overleaf.com/learn/latex/Learn_LaTeX_in_30_minutes
% https://en.wikibooks.org/wiki/LaTeX/Basics

% Formatting
\usepackage[utf8]{inputenc}
\usepackage[margin=1in]{geometry}
\usepackage[titletoc,title]{appendix}

% Math
% https://www.overleaf.com/learn/latex/Mathematical_expressions
% https://en.wikibooks.org/wiki/LaTeX/Mathematics
\usepackage{amsmath,amsfonts,amssymb,mathtools}

% Images
% https://www.overleaf.com/learn/latex/Inserting_Images
% https://en.wikibooks.org/wiki/LaTeX/Floats,_Figures_and_Captions
\usepackage{graphicx,float}

% Tables
% https://www.overleaf.com/learn/latex/Tables
% https://en.wikibooks.org/wiki/LaTeX/Tables


% Algorithms
% https://www.overleaf.com/learn/latex/algorithms
% https://en.wikibooks.org/wiki/LaTeX/Algorithms
\usepackage[ruled, vlined]{algorithm2e}
\usepackage{algorithmic}

% Code syntax highlighting
% https://www.overleaf.com/learn/latex/Code_Highlighting_with_minted
% \usepackage{minted}
%\usemintedstyle{borland}

% References
% https://www.overleaf.com/learn/latex/Bibliography_management_in_LaTeX
% https://en.wikibooks.org/wiki/LaTeX/Bibliography_Management
\usepackage{biblatex}
\addbibresource{references.bib}

\usepackage{hyperref}
\hypersetup{
    colorlinks=true,
    linkcolor=blue,
    filecolor=magenta,      
    urlcolor=cyan,
}

% Title content
\title{AMATH 582 Homework 2: Gobar Transform and Spectrogram for Audio Analysis}
\author{Hongda Li}
\date{\today}

\begin{document}

\maketitle

% Abstract
\begin{abstract}
   In this paper, we are interested in applying the principal of Gobar filtering on real world audio data. Our objective is to utilize different filtering techniques using wavelets to extract out sounds for certain instrument in the audio data. In addition, we also discuss the best way of choosing parameters for Gobar filtering to get the best visualizations for the Spectrogram. The results we present are musical notes of the guitar solo and bass solo for 2 rock songs. 
\end{abstract}


\section{Introduction and Overview}
    
    \par\hspace{1.1em}
    The objective is to reproduce the musical scores for a clip taken from the start of the song ``\textit{Sweet Child O' Mine}"  by Guns N' Roses and a guitar solo from \textit``{Comfortably Numb}" by Pink Floyd. We deploy the technique of first using a Shannon Filter on the whole audio to filter out only the frequencies for the instruments that we are interested in, using common knowledge and some basics in music, and then we use Gobar transform to find the best temporal and frequencies resolutions. Finally, we present a way to visualize the spectrogram, and an algorithm to identify the notes with some basics in music theory. In addition, we also explored ways of truncating the audio data so that it runs more efficiently.

\section{Theoretical Background}
    \par\hspace{1.1em}
    The Gobar transform consider the additional usage of a kernel function in the time frequencies, preferably a function that has bounded L2 Norm over the real complex domain.
    \begin{equation*}\tag{1}\label{eqn:1}
        g_{\tau, w}(\tau) = \exp(i\omega \tau)g(\tau - t)
    \end{equation*}
    \par
    $\omega$ is the multiplier for the Fourier Kernel. $g$ is a bounded function for filtering signal near $t$. A specific case of the Gobar transform is called the Short Time Fourier Transform(STFT), where we consider a real symmetric matrix to be the kernel. Together with the function $f(\tau)$ representing the signal, we have: 
    \begin{equation*}\tag{2}\label{eqn:2}
        \widetilde{f_g}(t, w) = \int_{-\infty}^{\infty} f(\tau)g(\tau - t)\exp(-i\omega t) d\tau
    \end{equation*}
    \par
    Notice that, the variance under the time and frequencies domain are bounded by some constant. This means that, shortening the width of the kernel will reduce Frequency resolution while increasing the with will reduce time resolution. % Citation here! 
    The problem can be addressed with some oversampling to produce better visualization to compensate. In addition, we consider the usage of several wavelet function for filtering in the time and frequency domain of the signal. 
    \begin{equation*}\tag{wavelet.G}\label{eqn:wavelet.G}
        g(\tau; t, w):=\exp \left(
            -
            \left(
                \frac{\tau - t}{\frac{\sqrt{2}w}{2}}
            \right)^2
        \right)
    \end{equation*}
    \begin{equation*}\tag{wavelet.S}\label{eqn:wavelet.S}
        g(\tau; t, w):=
        \begin{cases}
            1 & |\tau - t| \le w/2 \\ 0 & |\tau - t| > w/2
        \end{cases}
    \end{equation*}
    Where each of the wavelet function (Will be referred to as filter in the text) has one time parameter $\tau$ as the input, and its location and with is controlled by the parameters: $t, w$. In the case of the Gaussian Filter, $w$ means a distribution with $2w$ standard deviation. For computational purposes, the filtering processing using wavelets are implemented as element-wise vector multiplications and the STFT is implemented via FFT with a for loop that slide the kernel through the domain.


\section{Algorithm Implementation and Development}
    \par\hspace{1.1em}
    Two core parts of the algorithm is the parameters for the STFT, the implementations of STFT and the conversion between time domain vector and the frequencies domain vector, other aspects of the algorithm will be discussed as we view the results. The whole procedure of the processing any given audio can be summarized by: 
    \begin{enumerate}
        \item[1.] Setting up the parameters for the spectrogram, choosing a section of the music to analyze.
        \item[2.] FFT on the data, filter out a range of frequencies that the instrument is plying in with the Shannon Filter. 
        \item[3.] Create the spectrogram, truncate it to get the best view for the musical scores, and find out the peaking frequencies. 
    \end{enumerate}
    \subsection{Time to Frequencies Domain Conversion}
        \par\hspace{1.1em}
        The time domain vector is by a vector with the same length as the number of floats in the audio data. As common knowledge, digital audio data usually are sampled $480000$ times second (Uncompressed m4a format). 
        \begin{algorithm}\label{alg:1}
            \begin{algorithmic}[1]
                \STATE{\textbf{Input: }TimeVec}
                \STATE{n := length(TimeVec); L := TimeVec(end) - TimeVec(1);}
                \IF{n is even}
                    \STATE{hz:=$\frac{2\pi}{L}[0: n/2 - 1, -n/2: -1]$}
                \ELSE
                    \STATE{hz:=$\frac{2\pi}{L}[0, 1:(n - 1)/2, -(n - 1)/2:-1]$}
                \ENDIF
                \STATE{hz:= $\frac{\text{hz}}{2\pi}$}
                \STATE{\textbf{Output: } FFTshift(hz)}
            \end{algorithmic}\caption{Converting Time Domain to Frequencies Domain}
        \end{algorithm}
        \par
        Notice that, depending on the parity of the width of the signal, we need to partition the corresponding frequencies domain differently. In the case of even number of partitions, we want $0$ appears at index $n/2-1$ after the fftshift, assuming index start with $0$, and we want $0$ to be at index $\lfloor n/2\rfloor$ after fftshift assuming index starts with $0$. Most importantly, we need to divide it by $2\pi$ to actually obtain the frequencies for the audio.
        
    \subsection{Creation of Spectrogram Using STFT}
        \par\hspace{1.1em}
        The next part is the creation of the spectrogram. We implement the width of the filter to be relative to the window's size, and the window's size is simply the total length divides by the number of partitions we want. This is convenient when we sync up the STFT with the bars of music, which helps us get better results and visualization.
        \begin{algorithm}\label{alg:2}
            \begin{algorithmic}[1]
                \STATE{\textbf{Input:} TimeVec: Tvec, Audio: A, RelWidth: W, Number of Partitions: N, WaveletFunction: g}
                \STATE{\textbf{Initialize: }SpectroMatrix}
                \STATE{dt := (Tvec(end) - Tvec(1))/n}
                \STATE{Hz := Input Tvec into \hyperref[alg:1]{Algorithm 1}}
                \FOR{II = 0:N - 1}
                    \STATE{t := Tvec(0) + II*dt}
                    \STATE{F := g(Tvec; Tvec(1) + II*dt + dt/2, W*dt)}
                    \STATE{Signal := F*A}
                    \STATE{Signal := fftshift(fft(Signal))}
                    \STATE{Signal := Filter out excessive frequencies using Hz vector.}
                    \STATE{SpecMatrix(:, II + 1) = Signal}
                    \STATE{SpecMatrix(:, II + 1) = Normalize the II + 1 th row of SpecMatrix, and put it into log scale}
                \ENDFOR
                \STATE{\textbf{Output: } Tvec, Hz, SpecMatrix}
            \end{algorithmic}\caption{Creating Spectrogram with STFT}
        \end{algorithm}
        \par
        This algorithm (\hyperref[alg:1]{Algorithm 2}) is generic for all audio and wavelet function: A, and $g$. The variable ``RelWidth'' represents the with of the wavelet relative to the with of the partition of the signal in time domain, this adds more flexibility for simple oversampling, in addition, if ``N $\times$ RelWidth": is set to be a constant, then the amount of oversampling is kept. 
        \par
        In addition, extra processing (line 12 in \hyperref[alg:2]{Algorithm 2}) is involved to make better presentation on the spectrogram, trimming of frequencies that is negative and outside the range of the instrument we are interested in, and present it on a lot scale to scale down the relative sizes of the frequencies constant making the spectrogram more invisible. 
    \subsection{Global Frequencies Filtering and Audio Truncations}
        \par\hspace{1.1em}
        By common knowledge, music are written in bars of notes in Chromatic Scale or the Diatonic Scales for pop songs (Diatonic Scale is in the Chromatic Scale). Each note in Chromatic Scale increments by a multiplier of $2^{1/12}$ in term of frequencies (A semi-tone). An instrument such as guitar usually operates inside a range of 3 octaves (12 Semi-tone per octave). 
        \par
        The reasoning of filtering out the frequencies for just the instrument makes us easier to identify particular instruments using the Spectrogram. Implementations wise this is achieved via applying a Shannon Filter to the FFT of an audio section we are interested in, and then Inverse Transform it back. The process acts as a band pass filter and it isolate one of the instruments from other instruments. 
        \par
        Audio is truncated either by seconds, or by BPM (Beats per minute), both are implemented. The reasoning is that, music repeats around a theme with some variations, and they are located in sections of bars. Truncating the music in terms of bars reduce loads for computations, it also makes better visualization for the spectrogram. 

\section{Computational Results}
    \subsection{GNR: Notes and Spectrogram}
    

\section{Summary and Conclusions}
   
% References
\printbibliography

% Appendices
\begin{appendices}

\section{MATLAB Functions}

    
\end{appendices}

\end{document}
